% Options for packages loaded elsewhere
\PassOptionsToPackage{unicode}{hyperref}
\PassOptionsToPackage{hyphens}{url}
%
\documentclass[
]{article}
\usepackage{amsmath,amssymb}
\usepackage{iftex}
\ifPDFTeX
  \usepackage[T1]{fontenc}
  \usepackage[utf8]{inputenc}
  \usepackage{textcomp} % provide euro and other symbols
\else % if luatex or xetex
  \usepackage{unicode-math} % this also loads fontspec
  \defaultfontfeatures{Scale=MatchLowercase}
  \defaultfontfeatures[\rmfamily]{Ligatures=TeX,Scale=1}
\fi
\usepackage{lmodern}
\ifPDFTeX\else
  % xetex/luatex font selection
\fi
% Use upquote if available, for straight quotes in verbatim environments
\IfFileExists{upquote.sty}{\usepackage{upquote}}{}
\IfFileExists{microtype.sty}{% use microtype if available
  \usepackage[]{microtype}
  \UseMicrotypeSet[protrusion]{basicmath} % disable protrusion for tt fonts
}{}
\makeatletter
\@ifundefined{KOMAClassName}{% if non-KOMA class
  \IfFileExists{parskip.sty}{%
    \usepackage{parskip}
  }{% else
    \setlength{\parindent}{0pt}
    \setlength{\parskip}{6pt plus 2pt minus 1pt}}
}{% if KOMA class
  \KOMAoptions{parskip=half}}
\makeatother
\usepackage{xcolor}
\usepackage[margin=1in]{geometry}
\usepackage{color}
\usepackage{fancyvrb}
\newcommand{\VerbBar}{|}
\newcommand{\VERB}{\Verb[commandchars=\\\{\}]}
\DefineVerbatimEnvironment{Highlighting}{Verbatim}{commandchars=\\\{\}}
% Add ',fontsize=\small' for more characters per line
\usepackage{framed}
\definecolor{shadecolor}{RGB}{248,248,248}
\newenvironment{Shaded}{\begin{snugshade}}{\end{snugshade}}
\newcommand{\AlertTok}[1]{\textcolor[rgb]{0.94,0.16,0.16}{#1}}
\newcommand{\AnnotationTok}[1]{\textcolor[rgb]{0.56,0.35,0.01}{\textbf{\textit{#1}}}}
\newcommand{\AttributeTok}[1]{\textcolor[rgb]{0.13,0.29,0.53}{#1}}
\newcommand{\BaseNTok}[1]{\textcolor[rgb]{0.00,0.00,0.81}{#1}}
\newcommand{\BuiltInTok}[1]{#1}
\newcommand{\CharTok}[1]{\textcolor[rgb]{0.31,0.60,0.02}{#1}}
\newcommand{\CommentTok}[1]{\textcolor[rgb]{0.56,0.35,0.01}{\textit{#1}}}
\newcommand{\CommentVarTok}[1]{\textcolor[rgb]{0.56,0.35,0.01}{\textbf{\textit{#1}}}}
\newcommand{\ConstantTok}[1]{\textcolor[rgb]{0.56,0.35,0.01}{#1}}
\newcommand{\ControlFlowTok}[1]{\textcolor[rgb]{0.13,0.29,0.53}{\textbf{#1}}}
\newcommand{\DataTypeTok}[1]{\textcolor[rgb]{0.13,0.29,0.53}{#1}}
\newcommand{\DecValTok}[1]{\textcolor[rgb]{0.00,0.00,0.81}{#1}}
\newcommand{\DocumentationTok}[1]{\textcolor[rgb]{0.56,0.35,0.01}{\textbf{\textit{#1}}}}
\newcommand{\ErrorTok}[1]{\textcolor[rgb]{0.64,0.00,0.00}{\textbf{#1}}}
\newcommand{\ExtensionTok}[1]{#1}
\newcommand{\FloatTok}[1]{\textcolor[rgb]{0.00,0.00,0.81}{#1}}
\newcommand{\FunctionTok}[1]{\textcolor[rgb]{0.13,0.29,0.53}{\textbf{#1}}}
\newcommand{\ImportTok}[1]{#1}
\newcommand{\InformationTok}[1]{\textcolor[rgb]{0.56,0.35,0.01}{\textbf{\textit{#1}}}}
\newcommand{\KeywordTok}[1]{\textcolor[rgb]{0.13,0.29,0.53}{\textbf{#1}}}
\newcommand{\NormalTok}[1]{#1}
\newcommand{\OperatorTok}[1]{\textcolor[rgb]{0.81,0.36,0.00}{\textbf{#1}}}
\newcommand{\OtherTok}[1]{\textcolor[rgb]{0.56,0.35,0.01}{#1}}
\newcommand{\PreprocessorTok}[1]{\textcolor[rgb]{0.56,0.35,0.01}{\textit{#1}}}
\newcommand{\RegionMarkerTok}[1]{#1}
\newcommand{\SpecialCharTok}[1]{\textcolor[rgb]{0.81,0.36,0.00}{\textbf{#1}}}
\newcommand{\SpecialStringTok}[1]{\textcolor[rgb]{0.31,0.60,0.02}{#1}}
\newcommand{\StringTok}[1]{\textcolor[rgb]{0.31,0.60,0.02}{#1}}
\newcommand{\VariableTok}[1]{\textcolor[rgb]{0.00,0.00,0.00}{#1}}
\newcommand{\VerbatimStringTok}[1]{\textcolor[rgb]{0.31,0.60,0.02}{#1}}
\newcommand{\WarningTok}[1]{\textcolor[rgb]{0.56,0.35,0.01}{\textbf{\textit{#1}}}}
\usepackage{graphicx}
\makeatletter
\def\maxwidth{\ifdim\Gin@nat@width>\linewidth\linewidth\else\Gin@nat@width\fi}
\def\maxheight{\ifdim\Gin@nat@height>\textheight\textheight\else\Gin@nat@height\fi}
\makeatother
% Scale images if necessary, so that they will not overflow the page
% margins by default, and it is still possible to overwrite the defaults
% using explicit options in \includegraphics[width, height, ...]{}
\setkeys{Gin}{width=\maxwidth,height=\maxheight,keepaspectratio}
% Set default figure placement to htbp
\makeatletter
\def\fps@figure{htbp}
\makeatother
\setlength{\emergencystretch}{3em} % prevent overfull lines
\providecommand{\tightlist}{%
  \setlength{\itemsep}{0pt}\setlength{\parskip}{0pt}}
\setcounter{secnumdepth}{5}
\ifLuaTeX
  \usepackage{selnolig}  % disable illegal ligatures
\fi
\IfFileExists{bookmark.sty}{\usepackage{bookmark}}{\usepackage{hyperref}}
\IfFileExists{xurl.sty}{\usepackage{xurl}}{} % add URL line breaks if available
\urlstyle{same}
\hypersetup{
  pdftitle={Weather Data Analysis},
  hidelinks,
  pdfcreator={LaTeX via pandoc}}

\title{Weather Data Analysis}
\author{}
\date{\vspace{-2.5em}}

\begin{document}
\maketitle

\begin{center}\rule{0.5\linewidth}{0.5pt}\end{center}

\begin{center}\rule{0.5\linewidth}{0.5pt}\end{center}

\hypertarget{the-weather-dataset}{%
\section{The Weather Dataset}\label{the-weather-dataset}}

Here, The Weather Dataset is a time-series data set with per-hour
information about the weather conditions at a particular location. It
records Temperature, Dew Point Temperature, Relative Humidity, Wind
Speed, Visibility, Pressure, and Conditions.

This data is available as a CSV file.

\begin{verbatim}
## # A tibble: 8,784 x 8
##    `Date/Time`   Temp_C `Dew Point Temp_C` `Rel Hum_%` `Wind Speed_km/h`
##    <chr>          <dbl>              <dbl>       <dbl>             <dbl>
##  1 1/1/2012 0:00   -1.8               -3.9          86                 4
##  2 1/1/2012 1:00   -1.8               -3.7          87                 4
##  3 1/1/2012 2:00   -1.8               -3.4          89                 7
##  4 1/1/2012 3:00   -1.5               -3.2          88                 6
##  5 1/1/2012 4:00   -1.5               -3.3          88                 7
##  6 1/1/2012 5:00   -1.4               -3.3          87                 9
##  7 1/1/2012 6:00   -1.5               -3.1          89                 7
##  8 1/1/2012 7:00   -1.4               -3.6          85                 7
##  9 1/1/2012 8:00   -1.4               -3.6          85                 9
## 10 1/1/2012 9:00   -1.3               -3.1          88                15
## # i 8,774 more rows
## # i 3 more variables: Visibility_km <dbl>, Press_kPa <dbl>, Weather <chr>
\end{verbatim}

\hypertarget{head}{%
\subsection{.head()}\label{head}}

It shows the first N rows in the data (by default, N=5).

\begin{verbatim}
## # A tibble: 6 x 8
##   `Date/Time`   Temp_C `Dew Point Temp_C` `Rel Hum_%` `Wind Speed_km/h`
##   <chr>          <dbl>              <dbl>       <dbl>             <dbl>
## 1 1/1/2012 0:00   -1.8               -3.9          86                 4
## 2 1/1/2012 1:00   -1.8               -3.7          87                 4
## 3 1/1/2012 2:00   -1.8               -3.4          89                 7
## 4 1/1/2012 3:00   -1.5               -3.2          88                 6
## 5 1/1/2012 4:00   -1.5               -3.3          88                 7
## 6 1/1/2012 5:00   -1.4               -3.3          87                 9
## # i 3 more variables: Visibility_km <dbl>, Press_kPa <dbl>, Weather <chr>
\end{verbatim}

\hypertarget{shape}{%
\subsection{.shape}\label{shape}}

It shows the total no. of rows and no. of columns of the dataframe.

\begin{verbatim}
## (8784, 8)
\end{verbatim}

\hypertarget{index}{%
\subsection{.index}\label{index}}

This attribute provides the index of the dataframe.

\begin{verbatim}
## RangeIndex(start=0, stop=8784, step=1)
\end{verbatim}

\hypertarget{columns}{%
\subsection{.columns}\label{columns}}

It shows the name of each column.

\begin{verbatim}
## [1] "Date/Time"        "Temp_C"           "Dew Point Temp_C" "Rel Hum_%"       
## [5] "Wind Speed_km/h"  "Visibility_km"    "Press_kPa"        "Weather"
\end{verbatim}

\hypertarget{dtypes}{%
\subsection{.dtypes}\label{dtypes}}

It shows the data-type of each column.

\begin{verbatim}
## spc_tbl_ [8,784 x 8] (S3: spec_tbl_df/tbl_df/tbl/data.frame)
##  $ Date/Time       : chr [1:8784] "1/1/2012 0:00" "1/1/2012 1:00" "1/1/2012 2:00" "1/1/2012 3:00" ...
##  $ Temp_C          : num [1:8784] -1.8 -1.8 -1.8 -1.5 -1.5 -1.4 -1.5 -1.4 -1.4 -1.3 ...
##  $ Dew Point Temp_C: num [1:8784] -3.9 -3.7 -3.4 -3.2 -3.3 -3.3 -3.1 -3.6 -3.6 -3.1 ...
##  $ Rel Hum_%       : num [1:8784] 86 87 89 88 88 87 89 85 85 88 ...
##  $ Wind Speed_km/h : num [1:8784] 4 4 7 6 7 9 7 7 9 15 ...
##  $ Visibility_km   : num [1:8784] 8 8 4 4 4.8 6.4 6.4 8 8 4 ...
##  $ Press_kPa       : num [1:8784] 101 101 101 101 101 ...
##  $ Weather         : chr [1:8784] "Fog" "Fog" "Freezing Drizzle,Fog" "Freezing Drizzle,Fog" ...
##  - attr(*, "spec")=
##   .. cols(
##   ..   `Date/Time` = col_character(),
##   ..   Temp_C = col_double(),
##   ..   `Dew Point Temp_C` = col_double(),
##   ..   `Rel Hum_%` = col_double(),
##   ..   `Wind Speed_km/h` = col_double(),
##   ..   Visibility_km = col_double(),
##   ..   Press_kPa = col_double(),
##   ..   Weather = col_character()
##   .. )
##  - attr(*, "problems")=<externalptr>
\end{verbatim}

with python code the output is

\begin{verbatim}
## Date/Time            object
## Temp_C              float64
## Dew Point Temp_C    float64
## Rel Hum_%           float64
## Wind Speed_km/h     float64
## Visibility_km       float64
## Press_kPa           float64
## Weather              object
## dtype: object
\end{verbatim}

\hypertarget{unique}{%
\subsection{.unique()}\label{unique}}

In a column, it shows all the unique values. It can be applied on a
single column only, not on the whole dataframe.

\begin{verbatim}
##  [1] "Fog"                                    
##  [2] "Freezing Drizzle,Fog"                   
##  [3] "Mostly Cloudy"                          
##  [4] "Cloudy"                                 
##  [5] "Rain"                                   
##  [6] "Rain Showers"                           
##  [7] "Mainly Clear"                           
##  [8] "Snow Showers"                           
##  [9] "Snow"                                   
## [10] "Clear"                                  
## [11] "Freezing Rain,Fog"                      
## [12] "Freezing Rain"                          
## [13] "Freezing Drizzle"                       
## [14] "Rain,Snow"                              
## [15] "Moderate Snow"                          
## [16] "Freezing Drizzle,Snow"                  
## [17] "Freezing Rain,Snow Grains"              
## [18] "Snow,Blowing Snow"                      
## [19] "Freezing Fog"                           
## [20] "Haze"                                   
## [21] "Rain,Fog"                               
## [22] "Drizzle,Fog"                            
## [23] "Drizzle"                                
## [24] "Freezing Drizzle,Haze"                  
## [25] "Freezing Rain,Haze"                     
## [26] "Snow,Haze"                              
## [27] "Snow,Fog"                               
## [28] "Snow,Ice Pellets"                       
## [29] "Rain,Haze"                              
## [30] "Thunderstorms,Rain"                     
## [31] "Thunderstorms,Rain Showers"             
## [32] "Thunderstorms,Heavy Rain Showers"       
## [33] "Thunderstorms,Rain Showers,Fog"         
## [34] "Thunderstorms"                          
## [35] "Thunderstorms,Rain,Fog"                 
## [36] "Thunderstorms,Moderate Rain Showers,Fog"
## [37] "Rain Showers,Fog"                       
## [38] "Rain Showers,Snow Showers"              
## [39] "Snow Pellets"                           
## [40] "Rain,Snow,Fog"                          
## [41] "Moderate Rain,Fog"                      
## [42] "Freezing Rain,Ice Pellets,Fog"          
## [43] "Drizzle,Ice Pellets,Fog"                
## [44] "Drizzle,Snow"                           
## [45] "Rain,Ice Pellets"                       
## [46] "Drizzle,Snow,Fog"                       
## [47] "Rain,Snow Grains"                       
## [48] "Rain,Snow,Ice Pellets"                  
## [49] "Snow Showers,Fog"                       
## [50] "Moderate Snow,Blowing Snow"
\end{verbatim}

\hypertarget{is.na}{%
\subsection{.is.na}\label{is.na}}

SHow the total number of non-null Values in each column. It can be
applied in both the Dataframe and a single column

\begin{verbatim}
## [1] 0
\end{verbatim}

In a column, it shows all the unique values with their count. It can be
applied on single column only.

\begin{verbatim}
## Weather
## Mainly Clear                               2106
## Mostly Cloudy                              2069
## Cloudy                                     1728
## Clear                                      1326
## Snow                                        390
## Rain                                        306
## Rain Showers                                188
## Fog                                         150
## Rain,Fog                                    116
## Drizzle,Fog                                  80
## Snow Showers                                 60
## Drizzle                                      41
## Snow,Fog                                     37
## Snow,Blowing Snow                            19
## Rain,Snow                                    18
## Thunderstorms,Rain Showers                   16
## Haze                                         16
## Drizzle,Snow,Fog                             15
## Freezing Rain                                14
## Freezing Drizzle,Snow                        11
## Freezing Drizzle                              7
## Snow,Ice Pellets                              6
## Freezing Drizzle,Fog                          6
## Snow,Haze                                     5
## Freezing Fog                                  4
## Snow Showers,Fog                              4
## Moderate Snow                                 4
## Rain,Snow,Ice Pellets                         4
## Freezing Rain,Fog                             4
## Freezing Drizzle,Haze                         3
## Rain,Haze                                     3
## Thunderstorms,Rain                            3
## Thunderstorms,Rain Showers,Fog                3
## Freezing Rain,Haze                            2
## Drizzle,Snow                                  2
## Rain Showers,Snow Showers                     2
## Thunderstorms                                 2
## Moderate Snow,Blowing Snow                    2
## Rain Showers,Fog                              1
## Thunderstorms,Moderate Rain Showers,Fog       1
## Snow Pellets                                  1
## Rain,Snow,Fog                                 1
## Moderate Rain,Fog                             1
## Freezing Rain,Ice Pellets,Fog                 1
## Drizzle,Ice Pellets,Fog                       1
## Thunderstorms,Rain,Fog                        1
## Rain,Ice Pellets                              1
## Rain,Snow Grains                              1
## Thunderstorms,Heavy Rain Showers              1
## Freezing Rain,Snow Grains                     1
## Name: count, dtype: int64
\end{verbatim}

\hypertarget{count}{%
\subsection{.count}\label{count}}

\begin{verbatim}
## # A tibble: 1 x 1
##       n
##   <int>
## 1  8784
\end{verbatim}

\hypertarget{info}{%
\subsection{.info()}\label{info}}

Provides basic information about the dataframe.

\begin{verbatim}
##                  vars    n    mean      sd  median trimmed     mad    min
## Date/Time*          1 8784 4392.50 2535.87 4392.50 4392.50 3255.79   1.00
## Temp_C              2 8784    8.80   11.69    9.30    9.11   13.94 -23.30
## Dew Point Temp_C    3 8784    2.56   10.88    3.30    3.03   12.97 -28.50
## Rel Hum_%           4 8784   67.43   16.92   68.00   68.15   19.27  18.00
## Wind Speed_km/h     5 8784   14.95    8.69   13.00   14.27    8.90   0.00
## Visibility_km       6 8784   27.66   12.62   25.00   27.76    1.33   0.20
## Press_kPa           7 8784  101.05    0.84  101.07  101.07    0.76  97.52
## Weather*            8 8784   15.58   11.49   20.00   15.14    7.41   1.00
##                      max   range  skew kurtosis    se
## Date/Time*       8784.00 8783.00  0.00    -1.20 27.06
## Temp_C             33.00   56.30 -0.18    -0.92  0.12
## Dew Point Temp_C   24.40   52.90 -0.32    -0.82  0.12
## Rel Hum_%         100.00   82.00 -0.32    -0.55  0.18
## Wind Speed_km/h    83.00   83.00  0.87     1.54  0.09
## Visibility_km      48.30   48.10  0.41    -0.35  0.13
## Press_kPa         103.65    6.13 -0.23     0.71  0.01
## Weather*           50.00   49.00 -0.03    -1.14  0.12
\end{verbatim}

Lets Dive into Answering Some Useful Analysis Questions

\begin{center}\rule{0.5\linewidth}{0.5pt}\end{center}

\hypertarget{weather-analysis}{%
\section{Weather Analysis}\label{weather-analysis}}

\hypertarget{q-1.-find-all-the-unique-wind-speed-values-in-the-data.}{%
\subsection{Q) 1. Find all the unique `Wind Speed' values in the
data.}\label{q-1.-find-all-the-unique-wind-speed-values-in-the-data.}}

\begin{verbatim}
## # A tibble: 34 x 2
##    `Wind Speed_km/h`     n
##                <dbl> <int>
##  1                 0   309
##  2                 2     2
##  3                 4   474
##  4                 6   609
##  5                 7   677
##  6                 9   830
##  7                11   791
##  8                13   735
##  9                15   719
## 10                17   666
## # i 24 more rows
\end{verbatim}

Using Unique

\begin{verbatim}
##  [1]  4  7  6  9 15 13 20 22 19 24 30 35 39 32 33 26 44 43 48 37 28 17 11  0 83
## [26] 70 57 46 41 52 50 63 54  2
\end{verbatim}

\hypertarget{q-2.-find-the-number-of-times-when-the-weather-is-exactly-clear.}{%
\subsection{Q) 2. Find the number of times when the `Weather is exactly
Clear'.}\label{q-2.-find-the-number-of-times-when-the-weather-is-exactly-clear.}}

\begin{verbatim}
## # A tibble: 1 x 1
##       n
##   <int>
## 1  1326
\end{verbatim}

\begin{verbatim}
## # A tibble: 1,326 x 8
##    `Date/Time`    Temp_C `Dew Point Temp_C` `Rel Hum_%` `Wind Speed_km/h`
##    <chr>           <dbl>              <dbl>       <dbl>             <dbl>
##  1 1/3/2012 19:00  -16.9              -24.8          50                24
##  2 1/5/2012 18:00   -7.1              -14.4          56                11
##  3 1/5/2012 19:00   -9.2              -15.4          61                 7
##  4 1/5/2012 20:00   -9.8              -15.7          62                 9
##  5 1/5/2012 21:00   -9                -14.8          63                13
##  6 1/11/2012 1:00  -10.7              -17.8          56                17
##  7 1/11/2012 2:00  -12                -18.9          56                19
##  8 1/11/2012 3:00  -12.7              -19.4          57                19
##  9 1/11/2012 4:00  -13.4              -20.1          57                17
## 10 1/15/2012 8:00  -23.3              -28.5          62                 7
## # i 1,316 more rows
## # i 3 more variables: Visibility_km <dbl>, Press_kPa <dbl>, Weather <chr>
\end{verbatim}

\hypertarget{q-3.-find-the-number-of-times-when-the-wind-speed-was-exactly-4-kmh.}{%
\subsection{Q) 3. Find the number of times when the `Wind Speed was
exactly 4
km/h'.}\label{q-3.-find-the-number-of-times-when-the-wind-speed-was-exactly-4-kmh.}}

\begin{verbatim}
## # A tibble: 1 x 1
##       n
##   <int>
## 1   474
\end{verbatim}

\begin{verbatim}
## # A tibble: 474 x 8
##    `Date/Time`    Temp_C `Dew Point Temp_C` `Rel Hum_%` `Wind Speed_km/h`
##    <chr>           <dbl>              <dbl>       <dbl>             <dbl>
##  1 1/1/2012 0:00    -1.8               -3.9          86                 4
##  2 1/1/2012 1:00    -1.8               -3.7          87                 4
##  3 1/5/2012 0:00    -8.8              -11.7          79                 4
##  4 1/5/2012 5:00    -7                 -9.5          82                 4
##  5 1/7/2012 2:00    -8.1              -11.1          79                 4
##  6 1/7/2012 3:00    -7.8              -10.8          79                 4
##  7 1/7/2012 5:00    -6.9               -9.7          80                 4
##  8 1/7/2012 20:00   -1.8               -3.7          87                 4
##  9 1/7/2012 22:00   -1.5               -3            89                 4
## 10 1/9/2012 2:00    -9                -14.1          66                 4
## # i 464 more rows
## # i 3 more variables: Visibility_km <dbl>, Press_kPa <dbl>, Weather <chr>
\end{verbatim}

\hypertarget{q.-4-find-out-all-the-null-values-in-the-data.}{%
\subsection{Q. 4) Find out all the Null Values in the
data.}\label{q.-4-find-out-all-the-null-values-in-the-data.}}

\begin{verbatim}
## [1] FALSE
\end{verbatim}

This means that there are no null Values in the dataset \#\# Q. 5)
Rename the column name `Weather' of the dataframe to `Weather
Condition'.

\begin{verbatim}
## [1] "Date/Time"        "Temp_C"           "Dew Point Temp_C" "Rel Hum_%"       
## [5] "Wind Speed_km/h"  "Visibility_km"    "Press_kPa"        "Weather_Dataset"
\end{verbatim}

\hypertarget{q.6-what-is-the-mean-visibility}{%
\subsection{Q.6) What is the mean `Visibility'
?}\label{q.6-what-is-the-mean-visibility}}

\begin{verbatim}
##                  vars    n    mean      sd  median trimmed     mad    min
## Date/Time*          1 8784 4392.50 2535.87 4392.50 4392.50 3255.79   1.00
## Temp_C              2 8784    8.80   11.69    9.30    9.11   13.94 -23.30
## Dew Point Temp_C    3 8784    2.56   10.88    3.30    3.03   12.97 -28.50
## Rel Hum_%           4 8784   67.43   16.92   68.00   68.15   19.27  18.00
## Wind Speed_km/h     5 8784   14.95    8.69   13.00   14.27    8.90   0.00
## Visibility_km       6 8784   27.66   12.62   25.00   27.76    1.33   0.20
## Press_kPa           7 8784  101.05    0.84  101.07  101.07    0.76  97.52
## Weather*            8 8784   15.58   11.49   20.00   15.14    7.41   1.00
##                      max   range  skew kurtosis    se
## Date/Time*       8784.00 8783.00  0.00    -1.20 27.06
## Temp_C             33.00   56.30 -0.18    -0.92  0.12
## Dew Point Temp_C   24.40   52.90 -0.32    -0.82  0.12
## Rel Hum_%         100.00   82.00 -0.32    -0.55  0.18
## Wind Speed_km/h    83.00   83.00  0.87     1.54  0.09
## Visibility_km      48.30   48.10  0.41    -0.35  0.13
## Press_kPa         103.65    6.13 -0.23     0.71  0.01
## Weather*           50.00   49.00 -0.03    -1.14  0.12
\end{verbatim}

Singularly

\begin{verbatim}
## [1] 27.66445
\end{verbatim}

\hypertarget{q.-7-what-is-the-standard-deviation-of-pressure-in-this-data}{%
\subsection{Q. 7) What is the Standard Deviation of `Pressure' in this
data?}\label{q.-7-what-is-the-standard-deviation-of-pressure-in-this-data}}

\begin{verbatim}
## [1] 0.8440047
\end{verbatim}

\hypertarget{q.-8-whats-is-the-variance-of-relative-humidity-in-this-data}{%
\subsection{Q. 8) Whats is the Variance of `Relative Humidity' in this
data
?}\label{q.-8-whats-is-the-variance-of-relative-humidity-in-this-data}}

\begin{verbatim}
## [1] 286.2486
\end{verbatim}

\hypertarget{q.-9-find-all-instances-when-snow-was-recorded.}{%
\subsection{Q. 9) Find all instances when `Snow' was
recorded.}\label{q.-9-find-all-instances-when-snow-was-recorded.}}

looking for just the Instance ``Snow''

\begin{verbatim}
## # A tibble: 390 x 8
##    `Date/Time`    Temp_C `Dew Point Temp_C` `Rel Hum_%` `Wind Speed_km/h`
##    <chr>           <dbl>              <dbl>       <dbl>             <dbl>
##  1 1/3/2012 7:00   -14                -19.5          63                19
##  2 1/4/2012 12:00  -13.7              -21.7          51                11
##  3 1/4/2012 14:00  -11.3              -19            53                 7
##  4 1/4/2012 15:00  -10.2              -16.3          61                11
##  5 1/4/2012 16:00   -9.4              -15.5          61                13
##  6 1/4/2012 17:00   -8.9              -13.2          71                 9
##  7 1/4/2012 18:00   -8.9              -12.6          75                11
##  8 1/4/2012 19:00   -8.4              -12.7          71                 9
##  9 1/4/2012 20:00   -7.8              -12.1          71                 9
## 10 1/4/2012 21:00   -7.6              -11.6          73                 7
## # i 380 more rows
## # i 3 more variables: Visibility_km <dbl>, Press_kPa <dbl>, Weather <chr>
\end{verbatim}

But when we want to get all the columns that has ``Snow'' in it we use
\texttt{grepl} in R and \texttt{str.contains} in python

\begin{verbatim}
## # A tibble: 583 x 8
##    `Date/Time`    Temp_C `Dew Point Temp_C` `Rel Hum_%` `Wind Speed_km/h`
##    <chr>           <dbl>              <dbl>       <dbl>             <dbl>
##  1 1/2/2012 17:00   -2.1               -9.5          57                22
##  2 1/2/2012 20:00   -5.6              -13.4          54                24
##  3 1/2/2012 21:00   -5.8              -12.8          58                26
##  4 1/2/2012 23:00   -7.4              -14.1          59                17
##  5 1/3/2012 0:00    -9                -16            57                28
##  6 1/3/2012 2:00   -10.5              -15.8          65                22
##  7 1/3/2012 3:00   -11.3              -18.7          54                33
##  8 1/3/2012 5:00   -12.9              -19.1          60                22
##  9 1/3/2012 6:00   -13.3              -19.3          61                19
## 10 1/3/2012 7:00   -14                -19.5          63                19
## # i 573 more rows
## # i 3 more variables: Visibility_km <dbl>, Press_kPa <dbl>, Weather <chr>
\end{verbatim}

\hypertarget{q.-10-find-all-instances-when-wind-speed-is-above-24-and-visibility-is-25.}{%
\subsection{Q. 10) Find all instances when `Wind Speed is above 24' and
`Visibility is
25'.}\label{q.-10-find-all-instances-when-wind-speed-is-above-24-and-visibility-is-25.}}

\begin{verbatim}
## # A tibble: 3,324 x 8
##    `Date/Time`    Temp_C `Dew Point Temp_C` `Rel Hum_%` `Wind Speed_km/h`
##    <chr>           <dbl>              <dbl>       <dbl>             <dbl>
##  1 1/1/2012 20:00    3.2                1.3          87                19
##  2 1/1/2012 21:00    4                  1.7          85                20
##  3 1/1/2012 23:00    5.3                2            79                30
##  4 1/2/2012 0:00     5.2                1.5          77                35
##  5 1/2/2012 1:00     4.6                0            72                39
##  6 1/2/2012 2:00     3.9               -0.9          71                32
##  7 1/2/2012 3:00     3.7               -1.5          69                33
##  8 1/2/2012 4:00     2.9               -2.3          69                32
##  9 1/2/2012 5:00     2.6               -2.3          70                32
## 10 1/2/2012 6:00     2.3               -2.6          70                26
## # i 3,314 more rows
## # i 3 more variables: Visibility_km <dbl>, Press_kPa <dbl>, Weather <chr>
\end{verbatim}

\hypertarget{q.-11-what-is-the-mean-value-of-each-column-against-each-weather-conditon}{%
\subsection{Q. 11) What is the Mean value of each column against each
`Weather Conditon'
?}\label{q.-11-what-is-the-mean-value-of-each-column-against-each-weather-conditon}}

\begin{verbatim}
## # A tibble: 50 x 7
##    Weather Temp_C `Dew Point Temp_C` `Rel Hum_%` `Wind Speed_km/h` Visibility_km
##    <chr>    <dbl>              <dbl>       <dbl>             <dbl>         <dbl>
##  1 Clear    6.83              0.0894        64.5             10.6          30.2 
##  2 Cloudy   7.97              2.38          69.6             16.1          26.6 
##  3 Drizzle  7.35              5.50          88.2             16.1          17.9 
##  4 Drizzl~  8.07              7.03          93.3             11.9           5.26
##  5 Drizzl~  0.4              -0.7           92               20             4   
##  6 Drizzl~  1.05              0.15          93.5             14            10.5 
##  7 Drizzl~  0.693             0.12          95.9             15.5           5.51
##  8 Fog      4.30              3.16          92.3              7.95          6.25
##  9 Freezi~ -5.66             -8             83.6             16.6           9.2 
## 10 Freezi~ -2.53             -4.18          88.5             17             5.27
## # i 40 more rows
## # i 1 more variable: Press_kPa <dbl>
\end{verbatim}

Though the code is longer using R, Python makes It Easy for us

\begin{Shaded}
\begin{Highlighting}[]
\NormalTok{r.WeatherDataset.drop(}\StringTok{\textquotesingle{}Date/Time\textquotesingle{}}\NormalTok{, axis}\OperatorTok{=}\DecValTok{1}\NormalTok{).groupby(}\StringTok{\textquotesingle{}Weather\textquotesingle{}}\NormalTok{).mean()}
\end{Highlighting}
\end{Shaded}

\begin{verbatim}
##                                             Temp_C  ...   Press_kPa
## Weather                                             ...            
## Clear                                     6.825716  ...  101.587443
## Cloudy                                    7.970544  ...  100.911441
## Drizzle                                   7.353659  ...  100.435366
## Drizzle,Fog                               8.067500  ...  100.786625
## Drizzle,Ice Pellets,Fog                   0.400000  ...  100.790000
## Drizzle,Snow                              1.050000  ...  100.890000
## Drizzle,Snow,Fog                          0.693333  ...   99.281333
## Fog                                       4.303333  ...  101.184067
## Freezing Drizzle                         -5.657143  ...  100.202857
## Freezing Drizzle,Fog                     -2.533333  ...  100.441667
## Freezing Drizzle,Haze                    -5.433333  ...  100.316667
## Freezing Drizzle,Snow                    -5.109091  ...  100.520909
## Freezing Fog                             -7.575000  ...  102.320000
## Freezing Rain                            -3.885714  ...   99.647143
## Freezing Rain,Fog                        -2.225000  ...   99.945000
## Freezing Rain,Haze                       -4.900000  ...  100.375000
## Freezing Rain,Ice Pellets,Fog            -2.600000  ...  100.950000
## Freezing Rain,Snow Grains                -5.000000  ...   98.560000
## Haze                                     -0.200000  ...  101.482500
## Mainly Clear                             12.558927  ...  101.248832
## Moderate Rain,Fog                         1.700000  ...   99.980000
## Moderate Snow                            -5.525000  ...  100.275000
## Moderate Snow,Blowing Snow               -5.450000  ...  100.570000
## Mostly Cloudy                            10.574287  ...  101.025288
## Rain                                      9.786275  ...  100.233333
## Rain Showers                             13.722340  ...  100.404043
## Rain Showers,Fog                         12.800000  ...   99.830000
## Rain Showers,Snow Showers                 2.150000  ...  101.100000
## Rain,Fog                                  8.273276  ...  100.500862
## Rain,Haze                                 4.633333  ...  100.540000
## Rain,Ice Pellets                          0.600000  ...  100.120000
## Rain,Snow                                 1.055556  ...   99.951111
## Rain,Snow Grains                          1.900000  ...  100.600000
## Rain,Snow,Fog                             0.800000  ...  100.730000
## Rain,Snow,Ice Pellets                     1.100000  ...  100.105000
## Snow                                     -4.524103  ...  100.536103
## Snow Pellets                              0.700000  ...   99.700000
## Snow Showers                             -3.506667  ...  100.963500
## Snow Showers,Fog                        -10.675000  ...  101.292500
## Snow,Blowing Snow                        -5.410526  ...   99.704737
## Snow,Fog                                 -5.075676  ...  100.688649
## Snow,Haze                                -4.020000  ...  100.782000
## Snow,Ice Pellets                         -1.883333  ...  100.548333
## Thunderstorms                            24.150000  ...  100.230000
## Thunderstorms,Heavy Rain Showers         10.900000  ...  100.260000
## Thunderstorms,Moderate Rain Showers,Fog  19.600000  ...  100.010000
## Thunderstorms,Rain                       20.433333  ...  100.420000
## Thunderstorms,Rain Showers               20.037500  ...  100.233750
## Thunderstorms,Rain Showers,Fog           21.600000  ...  100.063333
## Thunderstorms,Rain,Fog                   20.600000  ...  100.080000
## 
## [50 rows x 6 columns]
\end{verbatim}

The Date/Time column is recorded as a calculated column, so we hav to
drop it \#\# Q. 12) What is the Minimum \& Maximum value of each column
against each `Weather Conditon' ?

\begin{verbatim}
## # A tibble: 50 x 7
##    Weather Temp_C `Dew Point Temp_C` `Rel Hum_%` `Wind Speed_km/h` Visibility_km
##    <chr>    <dbl>              <dbl>       <dbl>             <dbl>         <dbl>
##  1 Clear    -23.3              -28.5          20                 0          11.3
##  2 Cloudy   -21.4              -26.8          18                 0          11.3
##  3 Drizzle    1.1               -0.2          74                 0           6.4
##  4 Drizzl~    0                 -1.6          85                 0           1  
##  5 Drizzl~    0.4               -0.7          92                20           4  
##  6 Drizzl~    0.9                0.1          92                 9           9.7
##  7 Drizzl~    0.3               -0.1          92                 7           2.4
##  8 Fog      -16                -17.2          80                 0           0.2
##  9 Freezi~   -9                -12.2          78                 6           4.8
## 10 Freezi~   -6.4               -9            82                 6           3.6
## # i 40 more rows
## # i 1 more variable: Press_kPa <dbl>
\end{verbatim}

The Above is for the Minimun, The Maximum numbers include

\begin{verbatim}
## # A tibble: 50 x 7
##    Weather Temp_C `Dew Point Temp_C` `Rel Hum_%` `Wind Speed_km/h` Visibility_km
##    <chr>    <dbl>              <dbl>       <dbl>             <dbl>         <dbl>
##  1 Clear     32.8               20.4          99                33          48.3
##  2 Cloudy    30.5               22.6          99                54          48.3
##  3 Drizzle   18.8               17.7          96                30          25  
##  4 Drizzl~   19.9               19.1         100                28           9.7
##  5 Drizzl~    0.4               -0.7          92                20           4  
##  6 Drizzl~    1.2                0.2          95                19          11.3
##  7 Drizzl~    1.1                0.6          98                32           9.7
##  8 Fog       20.8               19.6         100                22           9.7
##  9 Freezi~   -2.3               -3.3          93                26          12.9
## 10 Freezi~   -0.3               -2.3          94                33           8  
## # i 40 more rows
## # i 1 more variable: Press_kPa <dbl>
\end{verbatim}

Lets try python code

\begin{Shaded}
\begin{Highlighting}[]
\NormalTok{r.WeatherDataset.groupby(}\StringTok{\textquotesingle{}Weather\textquotesingle{}}\NormalTok{).}\BuiltInTok{min}\NormalTok{()}
\end{Highlighting}
\end{Shaded}

\begin{verbatim}
##                                                 Date/Time  ...  Press_kPa
## Weather                                                    ...           
## Clear                                      1/11/2012 1:00  ...      99.52
## Cloudy                                     1/1/2012 17:00  ...      98.39
## Drizzle                                   1/23/2012 21:00  ...      97.84
## Drizzle,Fog                               1/23/2012 20:00  ...      98.65
## Drizzle,Ice Pellets,Fog                   12/17/2012 9:00  ...     100.79
## Drizzle,Snow                             12/17/2012 15:00  ...     100.63
## Drizzle,Snow,Fog                         12/18/2012 21:00  ...      97.79
## Fog                                         1/1/2012 0:00  ...      98.31
## Freezing Drizzle                          1/13/2012 10:00  ...      98.44
## Freezing Drizzle,Fog                        1/1/2012 2:00  ...      98.74
## Freezing Drizzle,Haze                      2/1/2012 11:00  ...     100.28
## Freezing Drizzle,Snow                      1/13/2012 3:00  ...      99.19
## Freezing Fog                               1/22/2012 6:00  ...     101.97
## Freezing Rain                             1/13/2012 11:00  ...      98.22
## Freezing Rain,Fog                         1/17/2012 23:00  ...      98.32
## Freezing Rain,Haze                         2/1/2012 14:00  ...     100.34
## Freezing Rain,Ice Pellets,Fog             12/17/2012 3:00  ...     100.95
## Freezing Rain,Snow Grains                  1/13/2012 9:00  ...      98.56
## Haze                                      1/22/2012 12:00  ...     100.35
## Mainly Clear                              1/10/2012 11:00  ...      98.67
## Moderate Rain,Fog                         12/10/2012 8:00  ...      99.98
## Moderate Snow                             1/12/2012 15:00  ...      99.88
## Moderate Snow,Blowing Snow               12/27/2012 10:00  ...     100.50
## Mostly Cloudy                              1/1/2012 16:00  ...      98.36
## Rain                                       1/1/2012 18:00  ...      97.52
## Rain Showers                               1/1/2012 22:00  ...      98.51
## Rain Showers,Fog                          10/20/2012 3:00  ...      99.83
## Rain Showers,Snow Showers                  11/4/2012 8:00  ...     101.09
## Rain,Fog                                  1/23/2012 18:00  ...      98.61
## Rain,Haze                                  3/13/2012 7:00  ...     100.50
## Rain,Ice Pellets                          12/18/2012 5:00  ...     100.12
## Rain,Snow                                  1/10/2012 5:00  ...      98.18
## Rain,Snow Grains                          12/21/2012 0:00  ...     100.60
## Rain,Snow,Fog                             12/8/2012 21:00  ...     100.73
## Rain,Snow,Ice Pellets                     12/21/2012 1:00  ...      99.85
## Snow                                       1/10/2012 1:00  ...      97.75
## Snow Pellets                             11/24/2012 15:00  ...      99.70
## Snow Showers                               1/12/2012 7:00  ...      99.49
## Snow Showers,Fog                          12/26/2012 9:00  ...     100.63
## Snow,Blowing Snow                         1/13/2012 21:00  ...      98.11
## Snow,Fog                                 12/16/2012 15:00  ...      99.38
## Snow,Haze                                  2/1/2012 17:00  ...     100.61
## Snow,Ice Pellets                          12/10/2012 3:00  ...      99.40
## Thunderstorms                              7/16/2012 1:00  ...      99.84
## Thunderstorms,Heavy Rain Showers           5/29/2012 6:00  ...     100.26
## Thunderstorms,Moderate Rain Showers,Fog    7/17/2012 6:00  ...     100.01
## Thunderstorms,Rain                        5/25/2012 20:00  ...     100.19
## Thunderstorms,Rain Showers                5/29/2012 16:00  ...      99.65
## Thunderstorms,Rain Showers,Fog             6/29/2012 3:00  ...      99.71
## Thunderstorms,Rain,Fog                     7/17/2012 5:00  ...     100.08
## 
## [50 rows x 7 columns]
\end{verbatim}

\hypertarget{q.-13-show-all-the-records-where-weather-condition-is-fog.}{%
\subsection{Q. 13) Show all the Records where Weather Condition is
Fog.}\label{q.-13-show-all-the-records-where-weather-condition-is-fog.}}

\begin{Shaded}
\begin{Highlighting}[]
\NormalTok{WeatherDataset }\SpecialCharTok{|\textgreater{}} 
  \FunctionTok{filter}\NormalTok{(Weather }\SpecialCharTok{==} \StringTok{\textquotesingle{}Fog\textquotesingle{}}\NormalTok{)}
\end{Highlighting}
\end{Shaded}

\begin{verbatim}
## # A tibble: 150 x 8
##    `Date/Time`    Temp_C `Dew Point Temp_C` `Rel Hum_%` `Wind Speed_km/h`
##    <chr>           <dbl>              <dbl>       <dbl>             <dbl>
##  1 1/1/2012 0:00    -1.8               -3.9          86                 4
##  2 1/1/2012 1:00    -1.8               -3.7          87                 4
##  3 1/1/2012 4:00    -1.5               -3.3          88                 7
##  4 1/1/2012 5:00    -1.4               -3.3          87                 9
##  5 1/1/2012 6:00    -1.5               -3.1          89                 7
##  6 1/1/2012 7:00    -1.4               -3.6          85                 7
##  7 1/1/2012 8:00    -1.4               -3.6          85                 9
##  8 1/1/2012 9:00    -1.3               -3.1          88                15
##  9 1/1/2012 10:00   -1                 -2.3          91                 9
## 10 1/1/2012 11:00   -0.5               -2.1          89                 7
## # i 140 more rows
## # i 3 more variables: Visibility_km <dbl>, Press_kPa <dbl>, Weather <chr>
\end{verbatim}

\hypertarget{q.-14-find-all-instances-when-weather-is-clear-or-visibility-is-above-40.}{%
\subsection{Q. 14) Find all instances when `Weather is Clear' or
`Visibility is above
40'.}\label{q.-14-find-all-instances-when-weather-is-clear-or-visibility-is-above-40.}}

\begin{Shaded}
\begin{Highlighting}[]
\NormalTok{WeatherDataset }\SpecialCharTok{|\textgreater{}} 
  \FunctionTok{filter}\NormalTok{((Visibility\_km }\SpecialCharTok{\textgreater{}} \DecValTok{40}\NormalTok{) }\SpecialCharTok{\&} \FunctionTok{grepl}\NormalTok{(}\StringTok{\textquotesingle{}Clear\textquotesingle{}}\NormalTok{, Weather))}
\end{Highlighting}
\end{Shaded}

\begin{verbatim}
## # A tibble: 1,184 x 8
##    `Date/Time`     Temp_C `Dew Point Temp_C` `Rel Hum_%` `Wind Speed_km/h`
##    <chr>            <dbl>              <dbl>       <dbl>             <dbl>
##  1 1/5/2012 10:00    -6                -10            73                17
##  2 1/5/2012 11:00    -5.6              -10.2          70                22
##  3 1/5/2012 12:00    -4.7               -9.6          69                20
##  4 1/5/2012 13:00    -4.4               -9.7          66                26
##  5 1/5/2012 14:00    -5.1              -10.7          65                22
##  6 1/5/2012 15:00    -4.3              -12            55                26
##  7 1/14/2012 13:00  -17.1              -24.1          55                17
##  8 1/15/2012 9:00   -22.2              -27.8          60                 9
##  9 1/15/2012 10:00  -20.6              -26.8          58                 9
## 10 1/15/2012 11:00  -19.3              -26.1          55                 9
## # i 1,174 more rows
## # i 3 more variables: Visibility_km <dbl>, Press_kPa <dbl>, Weather <chr>
\end{verbatim}

\hypertarget{q.-15-find-all-instances-when}{%
\subsection{Q. 15) Find all instances when
:}\label{q.-15-find-all-instances-when}}

\hypertarget{a.-weather-is-clear-and-relative-humidity-is-greater-than-50}{%
\subsubsection{A. `Weather is Clear' and `Relative Humidity is greater
than
50'}\label{a.-weather-is-clear-and-relative-humidity-is-greater-than-50}}

\hypertarget{or}{%
\subsubsection{or}\label{or}}

\hypertarget{b.-visibility-is-above-40}{%
\subsubsection{B. `Visibility is above
40'}\label{b.-visibility-is-above-40}}

\begin{verbatim}
## # A tibble: 4,034 x 8
##    `Date/Time`    Temp_C `Dew Point Temp_C` `Rel Hum_%` `Wind Speed_km/h`
##    <chr>           <dbl>              <dbl>       <dbl>             <dbl>
##  1 1/2/2012 12:00    1.7               -6.2          56                48
##  2 1/3/2012 12:00  -14.9              -22.6          52                20
##  3 1/3/2012 13:00  -15.1              -22.4          54                22
##  4 1/3/2012 15:00  -14.8              -22.2          53                19
##  5 1/3/2012 16:00  -15.3              -22.9          52                22
##  6 1/3/2012 17:00  -15.8              -23.2          53                22
##  7 1/3/2012 18:00  -16.3              -23.8          52                24
##  8 1/4/2012 1:00   -17.9              -24.1          58                11
##  9 1/4/2012 2:00   -18.1              -23.8          61                15
## 10 1/4/2012 3:00   -18.5              -24.6          59                13
## # i 4,024 more rows
## # i 3 more variables: Visibility_km <dbl>, Press_kPa <dbl>, Weather <chr>
\end{verbatim}

We have come to the End of the Descriptive Analysis.

By Precious Ikebude.

\begin{center}\rule{0.5\linewidth}{0.5pt}\end{center}

\end{document}
